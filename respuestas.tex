\documentclass{udparticle}
\setlogo{EIT}
\headertext{Anexo}
\title{Anexo}
\author{Thomas Muñoz, Dagoberto Navarrete.}
\usepackage{graphicx}
\usepackage{float}
\graphicspath{ {img/} }
\begin{document}
\maketitle
\begin{enumerate}
    \item Preguntas Conocimiento.\\
        \begin{enumerate}
            \item ¿Qué conceptos aprendí haciendo esta tarea? ¿Con qué otros conceptos los relaciono? \\
                {\bf Dagoberto}:\\
                {\bf Thomas}:Aprendí a identificar problemas que requieren el uso de {\textit Dynamic Programming} para ejecutarse, ya que de otro modo podría pasar horas o días sin entregar una respuesta (e.g. Fibonacci). Lo relaciono a los algoritmos utilizados para encontrar caminos en grafos.
            \item ¿Qué conozco del tema? ¿Qué conceptos me falta por aprender?\\
                {\bf Dagoberto}: \\
                {\bf Thomas}: Conozco como llegar a la subestructura óptima, me falta poder aterrizar estas ideas para implementar la estrategia en menos tiempo.
            \item ¿Qué conclusiones puedo sacar de este trabajo?\\
                {\bf Dagoberto}:\\
                {\bf Thomas}: Es un ejercicio interesante, ya que se aplica en algo que si bien sabemos utilizar no sabíamos como funcionaba, me permitió saber lo importante que es la programación dinámica y que los contenidos vistos en clases son los mismos que se utilizan en el área de la investigación.
        \end{enumerate}
    \item Preguntas sobre el proceso:\\
        \begin{itemize}
            \item ¿Qué habilidades trabajé haciendo esta tarea?\\
                {\bf Dagoberto}:\\
                {\bf Thomas}: Detección de subproblemas y memoización.
            \item ¿Qué pasos debo seguir para resolver un problema algorítmico?\\
                {\bf Dagoberto}: \\
                {\bf Thomas}: Identificar el problema, buscar una estructura óptima para la resolución de este y finalmente implementarla.
            \item ¿Qué pasos requirieron más tiempo? ¿Cuánto comprendí las instrucciones?\\
                {\bf Dagoberto}:\\ 
                {\bf Thomas}: Lo que más tiempo me tomó fue el leer libros para comprender el objetivo de la programación dinámica y su implementación.
        \end{itemize}
    \item Preguntas sobre las actitudes:\\
        \begin{itemize}
            \item ¿Cuán sistemático he sido? ¿Cuánto interés tengo en resolver la tarea?\\
                {\bf Dagoberto}: \\
                {\bf Thomas}: Los días viernes, el fin de semana y las tardes las dediqué a revisar el problema y/o leer los lbros del curso. El área de la investigación me interesa mucho, asi que estuve muy motivado en poder llegar a resolver la tarea.
            \item ¿Le he dedicado suficiente atención y concentración a la resolución de la tarea? \\
                {\bf Dagoberto}:\\
                {\bf Thomas}: Siento que podría haber dedicado más tiempo para internalizar la materia.
            \item ¿Colaboré con mis compañeros? ¿Qué rol asumo cuando trabajo en equipo?\\
                {\bf Dagoberto}: \\
                {\bf Thomas}: Si, creo que ambos asumimos el rol de suplir las falencias del otro, en el sentido de que quien se manejaba más en un tema dado le explicaba al otro como seguir.
        \end{itemize}
\end{enumerate}
\end{document}
